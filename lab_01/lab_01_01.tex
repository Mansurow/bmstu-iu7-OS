\documentclass[a4paper,14pt]{article}
\usepackage{amssymb}
\usepackage{amsmath}
\usepackage{amsthm}
\usepackage{caption}
\usepackage{misccorr}
\usepackage[noadjust]{cite}
\usepackage{cmap}
\usepackage[utf8x]{inputenc}
\usepackage[T2A]{fontenc}
\usepackage[english, russian]{babel}
\usepackage{graphics}
\usepackage{graphicx}
\usepackage{textcomp}
\usepackage{verbatim}
\usepackage{makeidx}
\usepackage{geometry}
\usepackage{float}
\usepackage{bm}
\usepackage{esint}
\usepackage{mathtools}
\usepackage{graphicx}
\usepackage{listings}
\usepackage{courier}
\usepackage{multirow}
\usepackage{graphicx}
\usepackage{xcolor}
\usepackage{ucs}
\usepackage{titlesec}

\lstdefinestyle{asm}{
	language={[x86masm]Assembler},
	backgroundcolor=\color{white},
	basicstyle=\footnotesize\ttfamily,
	keywordstyle=\color{blue},
	stringstyle=\color{red},
	commentstyle=\color{gray},
	numbers=left,
	numberstyle=\tiny,
	stepnumber=1,
	numbersep=5pt,
	frame=single,
	tabsize=4,
	captionpos=b,
	breaklines=true
}

\lstset{basicstyle=\fontsize{10}{10}\selectfont,breaklines=true,inputencoding=utf8x,extendedchars=\true}

\lstset{
	literate=
	{а}{{\selectfont\char224}}1
	{б}{{\selectfont\char225}}1
	{в}{{\selectfont\char226}}1
	{г}{{\selectfont\char227}}1
	{д}{{\selectfont\char228}}1
	{е}{{\selectfont\char229}}1
	{ё}{{\"e}}1
	{ж}{{\selectfont\char230}}1
	{з}{{\selectfont\char231}}1
	{и}{{\selectfont\char232}}1
	{й}{{\selectfont\char233}}1
	{к}{{\selectfont\char234}}1
	{л}{{\selectfont\char235}}1
	{м}{{\selectfont\char236}}1
	{н}{{\selectfont\char237}}1
	{о}{{\selectfont\char238}}1
	{п}{{\selectfont\char239}}1
	{р}{{\selectfont\char240}}1
	{с}{{\selectfont\char241}}1
	{т}{{\selectfont\char242}}1
	{у}{{\selectfont\char243}}1
	{ф}{{\selectfont\char244}}1
	{х}{{\selectfont\char245}}1
	{ц}{{\selectfont\char246}}1
	{ч}{{\selectfont\char247}}1
	{ш}{{\selectfont\char248}}1
	{щ}{{\selectfont\char249}}1
	{ъ}{{\selectfont\char250}}1
	{ы}{{\selectfont\char251}}1
	{ь}{{\selectfont\char252}}1
	{э}{{\selectfont\char253}}1
	{ю}{{\selectfont\char254}}1
	{я}{{\selectfont\char255}}1
	{А}{{\selectfont\char192}}1
	{Б}{{\selectfont\char193}}1
	{В}{{\selectfont\char194}}1
	{Г}{{\selectfont\char195}}1
	{Д}{{\selectfont\char196}}1
	{Е}{{\selectfont\char197}}1
	{Ё}{{\"E}}1
	{Ж}{{\selectfont\char198}}1
	{З}{{\selectfont\char199}}1
	{И}{{\selectfont\char200}}1
	{Й}{{\selectfont\char201}}1
	{К}{{\selectfont\char202}}1
	{Л}{{\selectfont\char203}}1
	{М}{{\selectfont\char204}}1
	{Н}{{\selectfont\char205}}1
	{О}{{\selectfont\char206}}1
	{П}{{\selectfont\char207}}1
	{Р}{{\selectfont\char208}}1
	{С}{{\selectfont\char209}}1
	{Т}{{\selectfont\char210}}1
	{У}{{\selectfont\char211}}1
	{Ф}{{\selectfont\char212}}1
	{Х}{{\selectfont\char213}}1
	{Ц}{{\selectfont\char214}}1
	{Ч}{{\selectfont\char215}}1
	{Ш}{{\selectfont\char216}}1
	{Щ}{{\selectfont\char217}}1
	{Ъ}{{\selectfont\char218}}1
	{Ы}{{\selectfont\char219}}1
	{Ь}{{\selectfont\char220}}1
	{Э}{{\selectfont\char221}}1
	{Ю}{{\selectfont\char222}}1
	{Я}{{\selectfont\char223}}1
}

%opening
\title{}
\author{Мансуров Владислав Михайлович}

% геометрия
\geometry{pdftex, left = 3cm, right = 10mm, top = 2cm, bottom = 2cm}

\setcounter{tocdepth}{4} % фикс переноса 
\righthyphenmin = 2
\tolerance = 2048

\begin{document}
	
\begin{titlepage}
	\noindent
	\begin{tabular}{|c|c|}	
	\hline
	\noindent	
	\begin{minipage}{0.14\textwidth}
		\includegraphics[width=\linewidth]{img/b_logo}
	\end{minipage} &
	\noindent
	\begin{minipage}{0.75\textwidth}\centering
		\textbf{\newline Министерство науки и высшего образования Российской Федерации}\\
		\textbf{Федеральное государственное бюджетное образовательное учреждение высшего образования}\\
		\textbf{«Московский государственный технический университет имени Н.Э.~Баумана}\\
		\textbf{(национальный исследовательский университет)»}\\
		\textbf{(МГТУ им. Н.Э.~Баумана)}
	\end{minipage} \\
	\hline	
	\end{tabular}
	\newline\newline
	\noindent ФАКУЛЬТЕТ \underline{«Информатика и системы управления»} \newline\newline
	\noindent КАФЕДРА \underline{«Программное обеспечение ЭВМ и информационные технологии»}\newline\newline\newline\newline\newline\newline\newline
	
	\begin{center}
		\noindent\begin{minipage}{1.0\textwidth}\centering
			\Large\textbf{   ~~~ Лабораторная работа №1}\newline
			\textbf{по дисциплине "Операционные системы"}\newline
		\end{minipage}
	\end{center}
	\noindent\textbf{Тема} $\underline{\text{Дизассемблирование INT 8h}}$\newline\newline
	\noindent\textbf{Студент} $\underline{\text{Мансуров В. М.}}$\newline\newline
	\noindent\textbf{Группа} $\underline{\text{ИУ7-56Б}}$\newline\newline
	\noindent\textbf{Преподаватель} $\underline{\text{Рязанова Н.Ю.}}$\newline
	
	\begin{center}
		\mbox{}
		\vfill
		Москва, \the\year ~г.
	\end{center}
	\clearpage
\end{titlepage}

\section{Полученный дизассемблированный код}
\subsection{Листинг обработчика прерывания INT 8} 
\begin{lstlisting}[style={asm}]
	 Temp.lst						 Sourcer Listing v3.07    11-Sep-22   9:18 pm 
	 Page 1	
;; Вызов подпрограммы sub_6:	
020A:0746  E8 0070				call	sub_6	; (07B9)

;; Сохранение значений регистров es, ds, ax, dx:
020A:0749  06					push	es
020A:074A  1E					push	ds
020A:074B  50					push	ax
020A:v074C  52					push	dx

;; Загрузка сегментных регистров ds, es: 
;; (40h - сегментная часть адреса области данных BIOS)
020A:074D  B8 0040				mov	ax,40h
020A:0750  8E D8				mov	ds,ax
020A:0752  33 C0				xor	ax,ax			
020A:0754  8E C0				mov	es,ax

;; Инкремент счётчиков таймера:
;; 0040:006C, 0040:006E - адреса младшего и старшего слова 
;; счётчика прерываний таймера BIOSа
020A:0756  FF 06 006C		    inc	word ptr ds:[6Ch]   	
;; (0040:006C=4E47h), по этому адресу располагается счетчик реального времен
020A:075A  75 04				jnz	loc_3			        ; Jump if not zero
020A:075C  FF 06 006E			inc	word ptr ds:[6Eh]	    ; (0040:006E=15h)

;; Сброс счётчиков времени при наступлении нового дня:
;; 0040:006E == 18h (24), 0040:006C == B0h (176)
;; 18h << 16 + B0h == 24 * 60 * 60 * c; 
;; c = 1573040 / 86400 = 18.2... - количество срабатываний таймера в секунду
;; Таким образом из того, что условие выполняется, следует, что прошли сутки.
020A:0760			      loc_3:
020A:0760  83 3E 006E 18		cmp	word ptr ds:[6Eh],18h	; (0040:006E=15h)
020A:0765  75 15				jne	loc_4			        ; Jump if not equal
020A:0767  81 3E 006C 00B0		cmp	word ptr ds:[6Ch],0B0h	; (0040:006C=4E47h)
020A:076D  75 0D				jne	loc_4			        ; Jump if not equal

;; Обнуление счетчика (старшего слова и младшего слова) таймера
020A:076F  A3 006E				mov	word ptr ds:[6Eh],ax	; (0040:006E=15h)
020A:0772  A3 006C				mov	word ptr ds:[6Ch],ax	; (0040:006C=4E47h)

;; Прошло более 24 часов, занесение значения 1 в 0040:0070
020A:0775  C6 06 0070 01		mov	byte ptr ds:[70h],1	    ; (0040:0070=0)
;; Установка al = 8:
020A:077A  0C 08				or	al,8

020A:077C			      loc_4:
;; Сохранение регистра ax:
020A:077C  50					push	ax
;; Декремент счётчика времени до отключения моторчика дисковода:
;; (0040:0040 - адрес счётчика времени в области данных накопителя FDD)
020A:077D  FE 0E 0040			dec	byte ptr ds:[40h]		; (0040:0040=96h)
020A:0781  75 0B				jnz	loc_5					; Jump if not zero
;; Установка флагов, отвечающих за отключение моторчикка дисковода:
020A:0783  80 26 003F F0		and	byte ptr ds:[3Fh],0F0h	; (0040:003F=0)
;; Отправка команды отключения моторчика дисковода:
020A:0788  B0 0C				mov	al,0Ch
020A:078A  BA 03F2				mov	dx,3F2h
020A:078D  EE					out	dx,al					; port 3F2h, dsk0 contrl output

020A:078E				 loc_5:
;; Восстановление регистра ax:
020A:078E  58					pop	ax
;; Проверка второго бита (Parity Flag - флаг чётности):
;; 0040:0314h - адрес области данных BIOS, содержащей копию флагов
020A:078F  F7 06 0314 0004		test	word ptr ds:[314h],4	; (0040:0314=3200h)
020A:0795  75 0C				jnz	loc_6						; Jump if not zero
;; Сохранение младшего байта регистра FLAGS в AH:
020A:0797  9F					lahf							; Load ah from flags
; Обмен значений регистров ah и al: 
; Теперь младший байт регистра FLAGS находится в младшем байте регистра ax
020A:0798  86 E0				xchg	ah,al
;; Сохранение регистра ax:
020A:079A  50					push	ax
;; Косвенный вызов пользовательского прерывания по адресу в таблице векторов прерываний:
;; В этом случае не произойдёт push регистра FLAGS, на его месте будет AX, 
;; который восстановится в регистр FLAGS после выхода из обработчика прерывания
020A:079B  26: FF 1E 0070		call	dword ptr es:[70h]		; (0000:0070=6ADh)
020A:07A0  EB 03				jmp	short loc_7					; (07A5)
020A:07A2  90					nop
;; Вызов пользовательского прерывания через int 1Ch:
020A:07A3				 loc_6:
020A:07A3  CD 1C				int	1Ch					; Timer break (call each 18.2ms)
;; Вызов подпрограммы sub_6:
020A:07A5				 loc_7:
020A:07A5  E8 0011				call	sub_6			; (07B9)
;; Сброс контроллера прерываний (отправка команды End Of Interrupt):
;; Разрешение обработки прерываний с текущим или более низким приоритетом
020A:07A8  B0 20				mov	al,20h				; ' '
020A:07AA  E6 20				out	20h,al				; port 20h, 8259-1 int command
													;  al = 20h, end of interrupt
;; Восстановление значений регистров es, ds, ax, dx:					
020A:07AC  5A					pop	dx
020A:07AD  58					pop	ax
020A:07AE  1F					pop	ds
020A:07AF  07					pop	es

020A:07B0  E9 FE99				jmp	$-164h
020A:07B3  C4					db	0C4h
020A:07B4  C4 0E 93E9			les	cx,dword ptr ds:[93E9h]	; (0000:93E9=3C0Ch) Load 32 bit ptr
020A:07B8  FE					db	0FEh
;; ...
020A:064C 1E   					push ds
020A:064D 50					push ax
;; ...
020A:0689 58 					pop ax
020A:068A 1F  					pop ds
;; Возврат из прерывания
020A:06AC CF					iret     ; Interrrupt return						
\end{lstlisting}
\clearpage
\subsection{Листинг процедуры sub\_6}
\begin{lstlisting}[style={asm}]
;			       SUBROUTINE
Temp.lst						 Sourcer Listing v3.07    11-Sep-22   9:18 pm   Page 2
				sub_6		proc	near
;; Сохранение значений регистров ds, ax:
020A:07B9  1E					push	ds
020A:07BA  50					push	ax
;; Загрузка сегментного регистра ds:
020A:07BB  B8 0040				mov	ax,40h
020A:07BE  8E D8				mov	ds,ax
;; Сохранение младшего байта регистра FLAGS в AH:
020A:07C0  9F					lahf									; Load ah from flags
;; Проверка DF и старшего бита IOPL по адресу 0040:0314h:
020A:07C1  F7 06 0314 2400		test	word ptr ds:[314h],2400h		; (0040:0314=3200h)
020A:07C7  75 0C				jnz	loc_9								; Jump if not zero
;; Обнуление 9 бита - сброс IF (запрет прерываний):
020A:07C9  F0> 81 26 0314 FDFF	lock	and	word ptr ds:[314h],0FDFFh	; (0040:0314=3200h)

020A:07D0			      loc_8:
;; Сохранение регистра AH в младший байт FLAGS:
020A:07D0  9E					sahf					; Store ah into flags
;; Восстановление значений регистров ds, ax:
020A:07D1  58					pop	ax
020A:07D2  1F					pop	ds

020A:07D3  EB 03				jmp	short loc_10		; (07D8)

020A:07D5			      loc_9:
;; Сброс IF ( Iterrupt flag )
020A:07D5  FA					cli					; Disable interrupts
020A:07D6  EB F8				jmp	short loc_8		; (07D0)

020A:07D8			     loc_10:
;; Возврат из подпрограммы:
020A:07D8  C3					retn
				sub_6		endp	
\end{lstlisting}
\clearpage
\section{Схема алгоритмов}
\subsection{Схема алгоритма обработчика INT8h}

\begin{flushright}
	\includegraphics[height=0.875\textheight]{img/int_8h-1.png}
	\clearpage
	\includegraphics[height=0.8\textheight]{img/int_8h-2.png}
	\includegraphics[height=0.8\textheight]{img/int_8h-3.png}
	\includegraphics[height=0.95\textheight]{img/int_8h-4.png}
\end{flushright}
\subsection{Схема алгоритма процедуры sub\_6}

\begin{center}
	\includegraphics[height=0.98\textheight]{img/int_8h-sub.png}
\end{center}
\end{document}
